\hypertarget{index_usage}{}\section{Basic Usage}\label{index_usage}
In order to use this static library, you need to compile first using make. You may also include the sources directly into your project and use them that way. \hypertarget{index_basic_info}{}\subsection{Information}\label{index_basic_info}
In order to use the data class, you need to ensure that your compilation creates a data.o or data.obj, depending on your setup. Next, to use it in your code, you need to include the \hyperlink{data_8h}{data.h} included. The data class does NOT depend on external libraries.\hypertarget{index_syntax_example}{}\subsection{Example}\label{index_syntax_example}
Using a list: 
\begin{DoxyCode}
 klop::Data data;//Note: Do not use () if you plan to init with null data
 data.toList();
 data.addChild(klop::Data(42));
 printf("Forty Two is %d\n",data[0].asInt());
\end{DoxyCode}
 Using keys 
\begin{DoxyCode}
 klop::Data data;//Note: Do not use () if you plan to init with null data
 data.toAssoc();
 data.addChild("F2",klop::Data(42));
 printf("Forty Two is %d\n",data["F2"].asInt());
\end{DoxyCode}
 Compounding nodes 
\begin{DoxyCode}
 klop::Data data;//Note: Do not use () if you plan to init with null data
 data.toAssoc();
 klop::Data data2;
 data2.toList();
 data2.addChild(klop::Data(42));
 data2.addChild(klop::Data("Forty Two as well"));
 data.addChild("F2",data2);
 printf("Forty Two is %d\n",data["F2"][0].asInt());
 printf("Forty Two is %s\n",data["F2"][1].asString());
\end{DoxyCode}
 \hypertarget{index_license}{}\section{License}\label{index_license}
Copyright (c) $<$2011$>$ $<$Levi (kloplop321.com)$>$

This software is provided 'as-\/is', without any express or implied warranty. In no event will the authors be held liable for any damages arising from the use of this software.

Permission is granted to anyone to use this software for any purpose, including commercial applications, and to alter it and redistribute it freely, subject to the following restrictions:

1. The origin of this software must not be misrepresented; you must not claim that you wrote the original software. If you use this software in a product, an acknowledgment in the product documentation would be appreciated but is not required.

2. Altered source versions must be plainly marked as such, and must not be misrepresented as being the original software.

3. This notice may not be removed or altered from any source distribution. 